\documentclass[12pt,utf8]{beamer}

% Gute Einführung zu LaTeX-Beamer: http://www2.informatik.hu-berlin.de/~mischulz/beamer.html

%-----PARAMETERS-----

%Wichtige Standard Pakete!
%\usepackage[german]{babel}
\usepackage{ngerman}
\usepackage{xcolor}
\usepackage{graphicx}
\usepackage{subcaption}
\usepackage{tikz}


%Für den Header notwendig!
%\usepackage[percent]{overpic}

\usepackage{hyperref} % für korrekte Links

%Einbinden des Themes
\input{design_latex-template/beamerthemeFOSSAG.sty}


%Standard Angaben
\title{
	\hspace*{8cm}
	\includegraphics[scale=0.2]{resources/logo_500px.png}
	\newline
	FOSS-AG
}
\subtitle{Git}
\author{@chef\_excellence}
\institute[FOSS AG]{\textbf{F}ree and \textbf{O}pen \textbf{S}ource \textbf{S}oftware \textbf{AG}}

\date{\today}

%-----IMPLEMENTATION-----
\begin{document}
	\begin{frame}
		\titlepage
	\end{frame}

	\begin{frame}
		\frametitle{Was ist Git?}
		\begin{itemize}
			\item Version Control System
			\item erlaubt verteilte Softwareentwicklung
		\end{itemize}
	\end{frame}

	\begin{frame}
		\begin{figure}
			\begin{subfigure}[t]{0.5\textwidth}
				\centering
				\includegraphics[scale=0.2]{resources/git.png}
				\caption*{\tiny{\cite{git_logo}}}
			\end{subfigure}
		\end{figure}
	\end{frame}
	
	% NOTES
	%
	% - Grundlegende Funktionen von Git
	%   - Commits
	%   - Branches
	%   - Forks
	%   - Merge
	%
	% - Git Workflow
	%   - Arbeiten auf verschiedenen Branches
	%   - Arbeiten mit Pull/Merge Request
	%   - Issue-Tracker
	%
	% - Demonstrieren am Beispiel von Gitlab 
	% 
	
	\begin{frame}
		\bibliographystyle{plain}
		\bibliography{literatur}
		\addcontentsline{toc}{section}{\bibname}
	\end{frame}
	
\end{document}
